% Zone-Tritone System: Formal Mathematical Framework
% Academic LaTeX Document
% Author: Greg Brown
% Date: December 27, 2025

\documentclass[12pt]{article}

% Required Packages
\usepackage{amsmath, amssymb, amsthm}
\usepackage{geometry}
\usepackage{hyperref}

% Page Setup
\geometry{letterpaper, margin=1in}

% Theorem Environments
\newtheorem{theorem}{Theorem}[section]
\newtheorem{proposition}[theorem]{Proposition}
\newtheorem{corollary}[theorem]{Corollary}
\newtheorem{lemma}[theorem]{Lemma}
\theoremstyle{definition}
\newtheorem{definition}[theorem]{Definition}

% Title Information
\title{Formal Framework of the Zone--Tritone System:\\
A Mathematical Account of Harmonic Gravity}
\author{Greg Brown}
\date{December 27, 2025}

\begin{document}

\maketitle

\begin{abstract}
We formalize the Zone--Tritone System as a harmonic framework expressed in the modular
pitch space $\mathbb{Z}_{12}$. Zones are defined as residue classes modulo $2$, tritones
as elements separated by $6\bmod 12$, and chromatic drift as translation operators on
tritone pairs. We prove that whole--step motion preserves zone identity, half--step
motion transfers energy between zones, and tritone drift generates perfect fourth
root motion. We also show that the melodic minor scale is dual--zone harmonic due to
the simultaneous presence of two tritone anchors. This yields a mathematically coherent
account of dominant function, chromaticism, and harmonic gravity.
\end{abstract}

\section{Formal Framework of the Zone--Tritone System}

\subsection{Pitch Space and Modular Arithmetic}

We represent the twelve pitch classes of equal temperament as the additive group
\[
\mathbb{Z}_{12}=\{0,1,2,\dots,11\}
\]
under addition modulo $12$.

A pitch class $p\in\mathbb{Z}_{12}$ corresponds canonically to:
\[
0=C,\;1=C\sharp,\;2=D,\;3=E\flat,\;4=E,\;
5=F,\;6=F\sharp,\;7=G,\;8=A\flat,\;9=A,\;10=B\flat,\;11=B.
\]

\subsection{Whole--Tone Zones}

Define the two whole--tone zones as the residue classes modulo $2$:
\[
Z_1=\{0,2,4,6,8,10\},\qquad Z_2=\{1,3,5,7,9,11\}.
\]

These satisfy:
\[
Z_1\cup Z_2 = \mathbb{Z}_{12},\qquad
Z_1\cap Z_2=\emptyset.
\]

\begin{proposition}
If $x\in Z_i$, then $x\pm 2 \in Z_i$.
\end{proposition}

\begin{proof}
If $x\equiv r \pmod{2}$ then $x\pm 2\equiv r\pmod{2}$.
Thus zone membership is preserved under addition of whole steps.
\end{proof}

We therefore interpret \emph{whole steps as zone--stabilizing motion}.

\subsection{Half--Step Motion}

A semitone is the transformation
\[
x\mapsto x\pm 1\pmod{12}.
\]

\begin{proposition}
If $x\in Z_1$ then $x\pm 1\in Z_2$, and conversely.
\end{proposition}

\begin{proof}
If $x\equiv 0\pmod{2}$ then $x\pm 1\equiv 1\pmod{2}$, and vice--versa.
\end{proof}

We therefore interpret \emph{semitones as zone--crossing transformations}.

\subsection{Tritones as Internal Zone Axes}

A tritone is defined as an unordered pair
\[
\tau = \{p,\;p+6\pmod{12}\}.
\]

\begin{proposition}
If $p\in Z_i$ then $p+6\in Z_i$.
\end{proposition}

\begin{proof}
Since $6\equiv 0\pmod{2}$, adding $6$ preserves parity.
\end{proof}

Thus every tritone belongs entirely to one zone and functions as an
\emph{internal axis of harmonic gravity}.

\section{Chromatic Tritone Drift and the Cycle of Fourths}

Let a tritone be denoted:
\[
\tau_n=\{p_n,\;p_n+6\}.
\]

Define chromatic drift by:
\[
\tau_{n+1}=\{p_n-1,\;p_n+6-1\}.
\]

In tonal practice, a tritone identifies a dominant seventh chord.
Let $G(\tau)$ denote the root of the dominant defined by $\tau$.
Then the following holds.

\begin{theorem}
Chromatic drift of a tritone induces a root motion of descending fourths:
\[
R_{n+1}\equiv R_n-5\pmod{12}.
\]
\end{theorem}

\begin{proof}
Two pitch classes separated by $6$ semitones determine the third and seventh
of a dominant chord. Shifting the pair down by a semitone preserves their
interval while lowering both voices.

The mapping from tritone to root is affine and therefore semitone--equivariant.
It follows that semitone descent in the tritone pair produces semitone descent
in the dominant root. In tonal convention, semitone descent of the third--seventh
pair corresponds to root descent by a perfect fourth, i.e.\ $-5\bmod 12$.
\end{proof}

Thus chromatic tritone motion and the classical dominant cycle in fourths
are structurally equivalent phenomena.

\section{Dual--Zone Structure of the Melodic Minor Scale}

Let $S\subset \mathbb{Z}_{12}$ denote a scale.
For $C$ melodic minor:
\[
S = \{0,2,3,5,7,9,11\}.
\]

\begin{proposition}
$S$ contains two tritone pairs:
\[
\tau_1=\{11,5\},\qquad \tau_2=\{3,9\}.
\]
\end{proposition}

\begin{proof}
Direct verification shows each unordered pair differs by $6\bmod 12$
and belongs to $S$.
\end{proof}

\begin{definition}
A pitch collection is said to be \emph{dual--zone harmonic} if it contains
at least two distinct tritone anchors.
\end{definition}

\begin{corollary}
The melodic minor scale is dual--zone harmonic.
\end{corollary}

\appendix

\section{Proof Appendix for the Zone--Tritone Canon}

\subsection{Axiom 1: Zones Define Color}
\textbf{Statement.}
Whole--tone zones are closed under whole--step motion.

\begin{proof}
Immediate from parity preservation modulo $2$.
\end{proof}

Thus remaining inside a zone preserves harmonic identity.

\subsection{Axiom 2: Tritones Define Gravity}
\textbf{Statement.}
Every tritone lies fully within a single zone.

\begin{proof}
Adding $6$ preserves parity, hence zone membership.
\end{proof}

Therefore the tritone is an internal symmetry axis.

\subsection{Axiom 3: Half--Steps Define Motion}
\textbf{Statement.}
Semitone motion changes zone membership.

\begin{proof}
Adding $1$ flips parity.
\end{proof}

Thus semitones are structural boundary crossings.

\subsection{Axiom 4: Chromatic Drift Produces Fourth Cycles}
\textbf{Statement.}
Chromatic translation of a tritone produces dominant roots descending by fourths.

\begin{proof}
Given the equivalence between the tritone and the $\{3,7\}$ structure of dominant harmony,
semitone descent of both tones preserves chord quality while lowering functional root
position by $5\bmod 12$.
\end{proof}

\subsection{Axiom 5: Melodic Minor as a Dual--Zone Hybrid}
\textbf{Statement.}
The melodic minor scale contains two independent tritone anchors.

\begin{proof}
Verified constructively for any transposition.
\end{proof}

\end{document}
