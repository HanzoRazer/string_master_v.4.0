\documentclass[12pt]{article}

\usepackage{amsmath, amssymb, amsthm}
\usepackage{geometry}
\usepackage{hyperref}
\usepackage{enumitem}

\geometry{margin=1in}

\title{The Zone--Tritone System:\\
A Unified Harmonic Framework in Modular Pitch Space}

\author{Greg Brown}
\date{\today}

\begin{document}
\maketitle

\begin{abstract}
We present the Zone--Tritone System, a harmonic framework defined in the
twelve-tone modular pitch space $\mathbb{Z}_{12}$. We show that the pitch
set partitions naturally into two whole--tone zones corresponding to parity
classes modulo~$2$. Whole--step motion preserves zone identity while semitone
motion crosses zone boundaries. Tritones are shown to be order--two elements
(i.e.\ involutions) lying fully within single zones, forming internal axes
of harmonic gravity. Chromatic translation of a tritone produces dominant
root motion descending by fourths. The melodic minor scale is proven to be
dual--zone harmonic, containing two independent tritone anchors, and a general
classification of harmonic systems by tritone rank is proposed. This yields
a coherent mathematical account of harmonic gravity, chromaticism, and modal
hybrid behavior.
\end{abstract}

\tableofcontents

\section{Introduction}

A central puzzle in tonal and post--tonal music theory is the interaction
between chromatic symmetry and functional harmonic direction. Jazz practice
demonstrates extensive use of tritone substitution, altered dominant sonorities,
and melodic minor derived harmonies, yet a compact formal model explaining
their internal relations has been lacking.

The Zone--Tritone System offers such a model. It is grounded in three structural
principles:

\begin{enumerate}[label=\textbf{Axiom \arabic*.}]
\item Whole--tone zones define harmonic color.
\item Tritones define harmonic gravity.
\item Half--steps define directional motion between zones.
\end{enumerate}

We formalize these axioms in $\mathbb{Z}_{12}$ and derive several consequences.

\section{Pitch Space}

\subsection{Definition of $\mathbb{Z}_{12}$}

We represent pitch classes as elements of the additive cyclic group
\[
\mathbb{Z}_{12}=\{0,1,\dots,11\}
\]
with arithmetic modulo~$12$.

We use the standard assignment
\[
0=C,\;1=C\sharp,\;2=D,\;3=E\flat,\;4=E,\;
5=F,\;6=F\sharp,\;7=G,\;8=A\flat,\;9=A,\;
10=B\flat,\;11=B.
\]

\section{Whole--Tone Zones}

\subsection{Definition}

Define
\[
Z_1=\{0,2,4,6,8,10\},\quad Z_2=\{1,3,5,7,9,11\}.
\]

\begin{theorem}[Zone Partition]
$Z_1$ and $Z_2$ partition $\mathbb{Z}_{12}$.
\end{theorem}

\begin{proof}
Every integer mod~$12$ is congruent to either $0$ or $1$ mod~$2$, establishing
disjointness and completeness.
\end{proof}

\subsection{Zone Stability Under Whole Steps}

\begin{proposition}
If $x\in Z_i$, then $x\pm 2\in Z_i$.
\end{proposition}

\begin{proof}
Parity is preserved.
\end{proof}

Thus whole--steps preserve harmonic color.

\section{Half--Step Motion as Zone Crossing}

\begin{proposition}
If $x\in Z_1$ then $x\pm 1\in Z_2$, and conversely.
\end{proposition}

\begin{proof}
Adding~$\pm 1$ flips parity.
\end{proof}

We interpret semitones as directional crossings between harmonic fields.

\section{Tritones as Order--Two Elements}

\subsection{Definition}

A tritone is the unordered pair
\[
\tau = \{p,\;p+6\}.
\]

\begin{theorem}
Every tritone lies within a single zone.
\end{theorem}

\begin{proof}
Since $6\equiv 0\pmod{2}$, adding six preserves parity.
\end{proof}

\subsection{Involution Property}

\[
(p+6)+6\equiv p\pmod{12}.
\]

Thus tritones are involutions and therefore symmetry axes.

\section{Chromatic Drift and the Cycle of Fourths}

Let $\tau_n=\{p_n,\;p_n+6\}$ and define
\[
\tau_{n+1}=\{p_n-1,\;p_n+5\}.
\]

Let $R_n$ denote the root of the associated dominant chord.

\begin{theorem}
Chromatic drift produces functional root motion by descending fourths:
\[
R_{n+1}\equiv R_n-5\pmod{12}.
\]
\end{theorem}

\begin{proof}
The tritone uniquely defines the third and seventh of a dominant chord. Their
chromatic descent corresponds to functional resolution by a fourth.
\end{proof}

\section{Dual--Zone Harmonic Systems}

\begin{definition}
A pitch collection is dual--zone harmonic if it contains at least two distinct
tritone anchors.
\end{definition}

\subsection{Melodic Minor}

For $C$ melodic minor,
\[
S=\{0,2,3,5,7,9,11\}.
\]

It contains:
\[
\{11,5\},\quad \{3,9\}.
\]

Thus melodic minor is dual--zone harmonic.

\section{Classification by Tritone Rank}

We define the \emph{tritone rank} of a collection as the number of distinct
tritone axes contained within it.

\begin{center}
\begin{tabular}{|c|c|c|}
\hline
System Type & Tritone Rank & Zone State\\
\hline
Major / Whole Tone & $1$ & Single-zone\\
Melodic Minor / Altered & $2$ & Dual-zone\\
Diminished & $4$ & Symmetry cloud\\
\hline
\end{tabular}
\end{center}

This provides a structural classification of harmonic environments.

\section{Worked Musical Examples}
\label{sec:examples}

In this section we illustrate the Zone--Tritone framework with concrete
harmonic progressions drawn from common-practice jazz vocabulary. For
clarity, we present chord symbols and pitch--class analysis; engraved
staff notation may be added in figures as needed.

\subsection{Example 1: Classical II--V--I in C Major}

Consider the progression
\[
\text{Dm7} \rightarrow \text{G7} \rightarrow \text{Cmaj7}.
\]

A common voicing in pitch classes is:
\begin{align*}
\text{Dm7} &: \{2,5,9,0\} \quad (D,F,A,C),\\
\text{G7}  &: \{7,11,2,5\} \quad (G,B,D,F),\\
\text{Cmaj7} &: \{0,4,7,11\} \quad (C,E,G,B).
\end{align*}

The tritone anchor of G7 is
\[
\tau = \{11,5\} = \{B,F\}.
\]

Both tones lie in the same zone:
\[
11\in Z_2,\quad 5\in Z_2.
\]

The voice--leading Dm7 $\rightarrow$ G7 $\rightarrow$ Cmaj7 is therefore
interpreted as:
\begin{enumerate}
\item Zone background: $Z_2$ stabilizes the tritone $\{11,5\}$.
\item Chromatic drift is not yet engaged; the tritone is static.
\item Resolution occurs when $\{11,5\}$ moves to $\{0,4,7\}$ (root and third
of Cmaj7), collapsing the gravity axis into a stable major triad.
\end{enumerate}

In practice, melodic material outlining $\{11,5\}$ over G7 aligns the
listener's ear with the active gravity axis.

\subsection{Example 2: Tritone Substitution}

Now replace G7 by its tritone substitute:
\[
\text{Dm7} \rightarrow \text{D}\flat\text{7} \rightarrow \text{Cmaj7}.
\]

A possible voicing:
\begin{align*}
\text{D}\flat\text{7} &: \{1,5,9,11\} \quad (D\flat,F,A\flat,B).
\end{align*}

The tritone anchor of D$\flat$7 is again
\[
\tau' = \{11,5\} = \{B,F\}.
\]

Thus both G7 and D$\flat$7 share the same tritone anchor and therefore the same
gravity axis, even though their roots differ by a tritone.

Within the Zone--Tritone System this is seen as an \emph{anchor exchange}
at the root level while the internal axis remains invariant.

\subsection{Example 3: Chromatic Dominant Chain}

Consider the chromatic sequence
\[
\text{G7} \rightarrow \text{C7} \rightarrow \text{F7} \rightarrow
\text{B}\flat\text{7}.
\]

Using tritone anchors:
\begin{align*}
\text{G7} &: \{11,5\} = \{B,F\},\\
\text{C7} &: \{10,4\} = \{B\flat,E\},\\
\text{F7} &: \{9,3\} = \{A,E\flat\},\\
\text{B}\flat\text{7} &: \{8,2\} = \{A\flat,D\}.
\end{align*}

Each step is a chromatic translation of the tritone by $-1$:
\[
\{11,5\} \rightarrow \{10,4\} \rightarrow \{9,3\} \rightarrow \{8,2\}.
\]

By Theorem~\ref{sec:examples}, this induces root motion by descending fourths:
\[
7 \rightarrow 0 \rightarrow 5 \rightarrow 10 \quad (\text{G} \rightarrow
\text{C} \rightarrow \text{F} \rightarrow \text{B}\flat).
\]

From a perceptual standpoint, the listener experiences a coherent
sequence of gravitational reorientations, despite the surface chromaticism.

\subsection{Example 4: Melodic Minor and Dual--Zone Harmony}

Consider $C$ melodic minor over a Cmi(maj7) sonority:
\[
\text{Cmi(maj7)}:\{0,3,7,11\} \quad (C,E\flat,G,B).
\]

The scale
\[
S = \{0,2,3,5,7,9,11\}
\]
contains two tritone anchors:
\[
\tau_1 = \{11,5\} = \{B,F\},\quad
\tau_2 = \{3,9\}  = \{E\flat,A\}.
\]

Improvisation that alternately emphasizes $\{11,5\}$ and $\{3,9\}$ engages
two independent gravity axes. This realizes the dual--zone nature of
melodic minor and explains the characteristic mixture of modal stability
and directed motion associated with this sound.

\subsection{Example 5: Minor II--V and Tritone Neighbors}

A typical minor II--V in C minor is
\[
\text{Dm7}\flat\text{5} \rightarrow \text{G7alt} \rightarrow \text{Cmi}.
\]

If G7alt is realized with a tritone anchor $\{11,5\}$ and extensive altered
tensions from the C altered scale, the Zone--Tritone model interprets the
alterations as controlled excursions within the dual--zone environment,
while the underlying gravity remains tied to the tritone axis.

Chromatic approach dominants (e.g.\ A$\flat$7 preceding G7alt) are treated as
intermediate states in a gravity chain, generated via tritone drift.

These examples illustrate that a wide range of jazz progressions can be
seen as specific realizations of zone stability, zone crossing, and tritone
drift, rather than as ad hoc chord--scale pairings.

\section{A Markov Model of Harmonic Gravity}

The linear recurrences derived in the Zone--Tritone System admit a natural
probabilistic generalization. Instead of treating harmonic motion as strictly
deterministic, we model it as a Markov process constrained by zone structure
and tritone anchors.

\subsection{State Space}

We define a discrete state space $\mathcal{S}$ whose elements may be chosen
at different levels of resolution:

\begin{itemize}
\item[\textbf{(i)}] \emph{Root states:} $\mathcal{S}_R=\mathbb{Z}_{12}$,
  where each state corresponds to a dominant root.
\item[\textbf{(ii)}] \emph{Axis states:} $\mathcal{S}_\tau=\{\tau_1,\dots,\tau_6\}$,
  the six tritone anchors.
\item[\textbf{(iii)}] \emph{Hybrid states:} pairs $(R,\tau)$
  with $\tau$ consistent with the dominant built on $R$.
\end{itemize}

For simplicity we first consider $\mathcal{S}_R$.

\subsection{Transition Kernel}

A first--order Markov chain on $\mathcal{S}_R$ is specified by a stochastic
matrix $P=(p_{ij})$ where
\[
p_{ij}=\Pr(R_{n+1}=j\mid R_n=i).
\]

In a purely functional idealization, we might set
\[
p_{i,j}=
\begin{cases}
1 & j\equiv i-5\ (\bmod 12),\\
0 & \text{otherwise},
\end{cases}
\]
thus recovering the strict cycle of fourths.

In practice, we allow a distribution concentrated around the functional
target:
\[
p_{i,j} =
\begin{cases}
\alpha & j\equiv i-5\ (\bmod 12)\\
\beta  & j\equiv i     \ (\bmod 12)\\
\gamma & j\equiv i+2   \ (\bmod 12)\\
\delta & \text{otherwise,}
\end{cases}
\]
with $\alpha>\beta\ge\gamma\ge\delta\ge 0$ and rows normalized to sum to~$1$.
Here $\alpha$ encodes functional resolution, while $\beta,\gamma,\delta$
encode local prolongation and neighboring motions.

\subsection{Zone--Conditioned Transitions}

We may refine the model by conditioning on zone structure. Let
$Z:\mathbb{Z}_{12}\to\{0,1\}$ be the zone map.

Define
\[
p_{i\to j} = f\bigl(Z(i),Z(j), C(i,j)\bigr)
\]
where $C(i,j)$ indicates zone crossing. For example, weights may be assigned so that:
\begin{itemize}
\item transitions preserving the active tritone axis but changing the root
      within a zone are given moderate probability,
\item transitions implementing chromatic tritone drift (and thus fourth
      motion) are given highest probability,
\item arbitrary jumps that disrupt both axis and zone receive low probability.
\end{itemize}

This yields a Markov chain whose high--probability paths correspond to
musically idiomatic motion.

\subsection{Axis--Based Markov Chain}

Alternatively, we may define a chain on the six tritone anchors
$\mathcal{S}_\tau$ by viewing chromatic drift as the primary transition:

\[
q_{kl}=
\begin{cases}
1 & \tau_l=T_{\pm1}(\tau_k)\\
0 & \text{otherwise,}
\end{cases}
\]
in the idealized case, or with softened probabilities in the empirical case.

This highlights the role of tritone movement as the principal driver of
functional reorientation.

\subsection{Applications}

Such Markov models have several applications:

\begin{enumerate}
\item \emph{Generative harmony:} sampling from the chain to produce idiomatic
      dominant progressions consistent with the Zone--Tritone grammar.
\item \emph{Style analysis:} estimating transition matrices from corpora of
      jazz standards or improvisations and comparing empirical statistics
      with theoretically predicted structure.
\item \emph{Pedagogical tools:} visualizing likely next states in a gravity
      chain as a student explores chord sequences.
\end{enumerate}

In all cases, the probabilistic model is constrained by the underlying
deterministic structure of zones, tritones, and chromatic drift.

\section{Methods: Estimating Gravity Transition Matrices from Jazz Repertoire}

To empirically validate the probabilistic gravity model introduced above,
we estimate transition matrices from annotated jazz harmony corpora. The
procedure described here allows reproducibility and comparison across
datasets and styles.

\subsection{Corpus Preparation}

Lead sheets, harmonic analyses, or symbolic encodings (e.g.\ MusicXML,
Humdrum, iRealPro, or hand-annotated chord charts) are converted into
ordered sequences of chord roots:
\[
(R_1, R_2, \dots, R_N), \qquad R_i \in \mathbb{Z}_{12}.
\]

Secondary dominants, tritone substitutions, backdoor dominants, and altered
dominants are normalized to dominant-function labels by taking the pitch
class of the chord root modulo $12$. Non-functional chords (e.g.\ sustained
modal sonorities) may either be excluded or treated as self-transitions,
depending on the study design.

\subsection{State Encoding}

Two levels of representation are supported:

\begin{enumerate}
\item \textbf{Root state model.} Each state corresponds to a dominant root
$R \in \mathbb{Z}_{12}$.

\item \textbf{Axis state model.} Each state corresponds to one of the six
tritone anchors
\[
\tau_k = \{p_k, p_k+6\}.
\]
The anchor associated with a dominant chord is taken to be its $(3,7)$ pair.
\end{enumerate}

Both representations yield Markov chains
\[
X_1, X_2, \dots, X_N
\]
on a finite state space.

\subsection{Counting Transitions}

For each adjacent pair $(X_i,X_{i+1})$ we increment a count
\[
c_{ab} \leftarrow c_{ab}+1
\qquad\text{whenever } X_i=a,\;X_{i+1}=b.
\]

This yields a raw transition count matrix
\[
C = (c_{ab}).
\]

To avoid zero-probability issues, Laplace smoothing may optionally be applied:
\[
\tilde c_{ab} = c_{ab} + \lambda,
\qquad \lambda \ge 0.
\]

\subsection{Normalizing to a Stochastic Matrix}

Row-wise normalization yields the empirical stochastic matrix
\[
p_{ab} = 
\frac{\tilde c_{ab}}{\sum_{b}\tilde c_{ab}}.
\]

Each row therefore encodes the conditional probability
\[
p_{ab} = \Pr(X_{n+1}=b \mid X_n=a).
\]

\subsection{Zone--Conditioned Analysis}

To assess the influence of whole--tone structure, each transition is also
classified by the zone map
\[
Z:\mathbb{Z}_{12}\to\{0,1\}.
\]

This yields empirical counts for:

\begin{itemize}
\item Zone-stable transitions ($Z(a)=Z(b)$),
\item Zone-crossing transitions ($Z(a)\neq Z(b)$),
\item Chromatic tritone drift transitions,
\[
\tau \rightarrow T_{\pm1}(\tau),
\]
\item Functional fourth motion
\[
R\rightarrow R-5 \pmod{12}.
\]
\end{itemize}

These categories allow statistical comparison between theoretical
predictions and observed practice.

\subsection{Goodness-of-Fit and Statistical Comparison}

Let $P_{\text{theory}}$ denote the idealized transition model and
$P_{\text{empirical}}$ the corpus-estimated matrix.

Similarity may be quantified using:
\begin{itemize}
\item Kullback--Leibler divergence,
\[
D_{\mathrm{KL}}(P_{\text{empirical}}\parallel P_{\text{theory}}),
\]
\item Matrix norm differences,
\item Eigenstructure comparison,
\item Stationary distribution analysis.
\end{itemize}

Significant concentration of probability mass around
descending-fourth motion and chromatic tritone drift supports the claim
that empirical jazz harmony conforms to Zone--Tritone gravity dynamics.

\subsection{Reproducibility}

All code, corpora, transition matrices, and statistical summaries should be
versioned and archived. We recommend symbolic encodings and open-source
analytics scripts to enable transparent replication and extension of this
work.

\section{Group--Theoretic Interpretation}

\subsection{Pitch Space as a Cyclic Group}

The pitch set $\mathbb{Z}_{12}$ forms a cyclic group under addition modulo~$12$.
We emphasize that all harmonic transformations in the Zone--Tritone System
are endomorphisms of this group.

\subsection{Parity Cosets as Whole--Tone Zones}

Define the subgroup
\[
2\mathbb{Z}_{12}=\{0,2,4,6,8,10\}.
\]
Then $Z_1 = 2\mathbb{Z}_{12}$ and $Z_2 = 1+2\mathbb{Z}_{12}$ are the two cosets.

\begin{theorem}
Whole--tone zones are precisely the cosets of the even--step subgroup.
\end{theorem}

\begin{proof}
Every element belongs to exactly one residue class modulo $2$.
\end{proof}

Thus zone membership is a group--theoretic parity property.

\subsection{Tritones as Order--Two Elements}

Consider the map
\[
\iota(x)=x+6.
\]

\begin{theorem}
$\iota$ is an involution:
\[
\iota\circ\iota=\text{id}.
\]
\end{theorem}

\begin{proof}
$(x+6)+6\equiv x\pmod{12}$.
\end{proof}

Hence every tritone axis corresponds to an element of order~$2$.

\subsection{Translation Symmetry and Chromatic Drift}

Define the translation operator
\[
T_k(x)=x+k.
\]

Then chromatic tritone drift is the orbit
\[
\tau_n = T_{-n}(\tau_0).
\]

\begin{theorem}
The set of tritone axes is invariant under the translation group
generated by $T_{\pm 1}$.
\end{theorem}

\begin{proof}
Translation preserves interval structure.
\end{proof}

This establishes chromatic drift as a symmetry action.

\subsection{Functional Harmony as Coupled Linear Dynamics}

Let $R_n$ denote the dominant root associated to $\tau_n$.
Then
\[
R_{n+1}\equiv R_n-5\pmod{12}
\]
is a linear recurrence relation.

Thus tritone drift and harmonic resolution correspond to coupled
linear actions on $\mathbb{Z}_{12}$, revealing functional harmony
as a symmetry--constrained dynamical process.

\section{Computational Formalization}

\subsection{Zone Indicator Function}

Define the zone map
\[
Z:\mathbb{Z}_{12}\rightarrow\{0,1\}
\]
by
\[
Z(p)=p\bmod 2.
\]

Then
\[
Z(p)=Z(q)\iff p-q\equiv 0\pmod{2}.
\]

\subsection{Crossing and Stability Operators}

Define the boolean functions
\[
C(p,q)=
\begin{cases}
1 & Z(p)\neq Z(q)\\
0 & Z(p)=Z(q)
\end{cases}
\]
and
\[
S(p,q)=1-C(p,q).
\]

Thus $C$ encodes directional motion.

\subsection{Tritone Detection Algorithm}

A pair $(p,q)$ forms a tritone iff
\[
(p-q)\equiv 6\pmod{12}.
\]

This yields a constant--time test suitable for streaming harmonic analysis.

\subsection{Dual--Zone Harmonic Detection}

Let $S\subset\mathbb{Z}_{12}$.

\begin{definition}
$S$ is dual--zone harmonic iff
\[
|\{\{p,p+6\} \subset S\}| \ge 2.
\]
\end{definition}

This allows automatic classification via computational search.

\subsection{Gravity Chain Generator}

Given an initial dominant root $R_0$, define
\[
R_{n+1}=R_n-5\pmod{12}.
\]

This produces the functional resolution chain algorithmically.
Such mappings may be implemented as deterministic automata
or extended probabilistically in Markov or Bayesian models.

\section{Psychoacoustic Interpretation}

\subsection{Symmetry, Ambiguity, and Suspension}

Highly symmetric pitch structures are perceptually associated with
reduced functional gravity. Whole--tone partitions are maximally
symmetric under $T_{\pm 2}$ translation, and listeners report a sense
of ``floating'' or non--directional color.

This supports the identification of zones with harmonic color fields.

\subsection{Tritones as Anchored Tension}

The tritone is maximally distant in $\mathbb{Z}_{12}$
and forms an internal reference axis. Psychoacoustic
studies have long associated the tritone with tension,
but within our model it is not random dissonance --- rather,
it is structured gravitational polarity.

\subsection{Semitone Crossings and Directional Energy}

Semitone motion is perceptually salient and strongly directional.
Since semitone movement crosses the zone boundary, the ear may be
interpreted as tracking transitions between symmetry fields.

Thus:
\begin{center}
perceived direction $\Longleftrightarrow$ zone crossing.
\end{center}

\subsection{Why Melodic Minor Feels ``Modern''}

Dual--zone harmonic systems support simultaneous stability and drift.
The coexistence of two tritone anchors allows:
\begin{itemize}
\item modal framing
\item functional implication
\item smooth chromatic weaving
\end{itemize}
without full collapse into tonal hierarchy.

This explains the distinctive affective quality of melodic minor harmony.

\subsection{Cognitive Coherence}

The Zone--Tritone System suggests that listeners balance two perceptual
forces:

\begin{enumerate}
\item symmetry--seeking (zone stabilization)
\item goal--seeking (tritone resolution)
\end{enumerate}

Music becomes meaningful in the tension between them.

\section{Relation to Prior Work}

Traditional jazz and tonal theory describe tritone substitution and
cycle--of--fourths resolution operationally. The present framework
derives these as structural consequences of modular symmetry and
translation invariance.

Thus the Zone--Tritone System may be viewed not as a replacement
for existing theory, but as its formal unifying grammar.

\section{Discussion}

The Zone--Tritone System resolves multiple theoretical tensions by showing that
functional direction arises from semitone-driven transitions between symmetry
fields, while gravitational identity arises from tritone axes internal to each
field.

\section{Conclusion}

We have demonstrated that whole--tone parity structure, tritone involutions,
and chromatic drift cohere into a compact formal grammar capable of explaining
a wide range of harmonic behavior in tonal and post--tonal music, particularly
within the jazz tradition.

\appendix
\section{Proof Appendix for Canonical Axioms}

\subsection{Axiom 1}
Closedness under whole steps is parity preservation.

\subsection{Axiom 2}
Tritones preserve parity and therefore zone identity.

\subsection{Axiom 3}
Half--steps flip parity.

\subsection{Axiom 4}
Chromatic drift corresponds to $-5\pmod{12}$ root motion.

\subsection{Axiom 5}
Melodic minor contains two independent tritone anchors.

\end{document}
